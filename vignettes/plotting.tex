%\VignetteIndexEntry{plotting}
\documentclass{article}

%%% latex packages
\usepackage{hyperref}
\usepackage[utf8]{inputenc}
\usepackage[usenames,dvipsnames]{xcolor}

\usepackage{lipsum} % for dummy text only

%% change margins to 1" all the way around
\oddsidemargin 0.0in
\evensidemargin 0.0in
\textwidth 6.5in
\headheight 0.0in
\topmargin 0.0in
\textheight 9.0in

%%% document info
\title{Plotting with \texttt{simPlot} in \texttt{SpaDES}}

\author{
  Alex M. Chubaty\\
  \small{Natural Resources Canada, Pacific Forestry Centre}\\
	\small{email: \href{mailto:achubaty@nrcan.gc.ca}{achubaty@nrcan.gc.ca}}
	\and
	Eliot McIntire\\
	\small{Natural Resources Canada, Pacific Forestry Centre}\\
	\small{email: \href{mailto:emcintir@nrcan.gc.ca}{emcintir@nrcan.gc.ca}}
}

\usepackage{Sweave}
\begin{document}
\Sconcordance{concordance:plotting.tex:plotting.Rnw:%
1 31 1 1 0 15 1}

 % displays code as entered (no arranging lines)

\maketitle

\tableofcontents

\newpage

\section{Plotting with \texttt{simPlot}}

\paragraph{}
One of the major features of the \texttt{SpaDES} package is that it brings together many of \textsf{R}'s powerful plotting and visualization into a small family of related functions. These new plotting functions make it easy to quickly produce visualizations useful at everystage of model development. Furthermore, conventional R plotting still works, so you can use the features provided in this package without having to relearn a completely new set of plotting commands.

\end{document}
