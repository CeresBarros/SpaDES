%\VignetteIndexEntry{Simulation visualization and plotting in SpaDES}
%\VignetteDepends{SpaDES}
%\VignetteKeyword{maps}
%\VignetteKeyword{plot}
%\VignetteKeyword{visualization}
\documentclass{article}

%%% latex packages
\usepackage{hyperref}
\usepackage[utf8]{inputenc}
\usepackage[usenames,dvipsnames]{xcolor}

%% change margins to 1" all the way around
\oddsidemargin 0.0in
\evensidemargin 0.0in
\textwidth 6.5in
\headheight 0.0in
\topmargin 0.0in
\textheight 9.0in

%%% document info
\title{Plotting with \texttt{Plot} in \texttt{SpaDES}}

\author{
	Eliot McIntire\\
	\small{Natural Resources Canada, Pacific Forestry Centre}\\
	\small{email: \href{mailto:emcintir@nrcan.gc.ca}{emcintir@nrcan.gc.ca}}
  \and
  Alex M. Chubaty\\
  \small{Natural Resources Canada, Pacific Forestry Centre}\\
  \small{email: \href{mailto:achubaty@nrcan.gc.ca}{achubaty@nrcan.gc.ca}}
}

\usepackage{Sweave}
\begin{document}
\Sconcordance{concordance:plotting.tex:plotting.Rnw:%
1 31 1 1 0 15 1}

 % displays code as entered (no arranging lines)

\maketitle

\tableofcontents

\newpage

\section{Plotting with \texttt{Plot}}

\paragraph{}
One of the major features of the \texttt{SpaDES} package is that it brings together many of \textsf{R}'s powerful plotting and visualization into a small family of related functions. These new plotting functions make it easy to quickly produce visualizations useful at every stage of model development. Furthermore, conventional R plotting still works, so you can use the features provided in this package or you can use base plotting functions without having to relearn a completely new set of plotting commands.


\begin{Schunk}
\begin{Sinput}
> library(raster)
> library(RColorBrewer)
> # Give dimensions of dummy raster
> nx <- 1e2
> ny <- 1e2
> template <- raster(nrows=ny, ncols=nx, xmn=-nx/2, xmx=nx/2, ymn =-ny/2, ymx=ny/2)
> # Make dummy maps for testing of models:
> # - digital elevation model (DEM)
> # - forest age
> # - forset cover
> # - percent pine
> DEM <- round(GaussMap(template, scale=300, var=0.03, speedup=1), 1)*1000
> forestAge <- round(GaussMap(template, scale=10, var=0.1, speedup=1), 1)*20
> forestCover <- round(GaussMap(template, scale=50, var=1, speedup=1),2)*10
> percentPine <- round(GaussMap(template, scale=50, var=1, speedup=1),1)
> # Scale them as needed
> forestAge <- forestAge/maxValue(forestAge)*100
> percentPine <- percentPine/maxValue(percentPine)*100
> # Make layers that are derived from other layers
> habitatQuality <- (DEM+10 + (forestCover+5)*10)/100
> habitatQuality <- habitatQuality/maxValue(habitatQuality)
> # Stack them into a single stack for plotting
> habitat <- stack(list(DEM, forestAge, forestCover, habitatQuality, percentPine))
> names(habitat) <- c("DEM", "forestAge", "forestCover", "habitatQuality", "percentPine")
> name(habitat) <- "habitat"
> Plot(habitat)
\end{Sinput}
\end{Schunk}

\begin{Schunk}
\begin{Sinput}
> nx <- 1e2
> ny <- 1e2
> template <- raster(nrows=ny, ncols=nx, xmn=-nx/2, xmx=nx/2, ymn =-ny/2, ymx=ny/2)